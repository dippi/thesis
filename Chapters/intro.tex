\chapter*{Introduction}
\addcontentsline{toc}{chapter}{\tocEntry{Introduction}}
\markboth{INTRODUCTION}{}

The exponential growth of data production, due to the increasing spread of social platforms and \emph{Internet of Things} (IoT), has started a thriving period of research and renovation in information technology. The relational model, that was once considered the silver bullet of data management, had reached its limits and many new paradigms have been proposed and developed.\\
While the initial phase was characterized by diversification and specialization, nowadays the progressive diffusion and stabilization of those tools is leading toward overlapping, interaction and unification of functionalities. In particular, in the field of \emph{Information Flow Processing} (IFP) \cite{ifp-survey} there is a growing interest in the interoperability with persistent data sources.

The project investigate the feasibility and limits of such an integration in the T-Rex CEP engine, showing that the new concepts can find a natural fit into the rule definition language and that the performance of an embedded database can satisfy the requirements of a real time execution, especially if in combination with a cache mechanism. However we highlighted how the reliability of the external component and the cache friendliness of the processed data remain strict requirements for a practical usability.

In particular my contribution were: identification of ambiguities in some \emph{TESLA} \cite{tesla} operators and proposal of refinements, supported by a formal definition of the syntax, that previously presented mostly by examples. Rewrite of the T-Rex engine to make it more robust, extensible and accurate with respect to the specification. Design and formalization of a syntactic and semantic extension. Development of the integration with \emph{SQLite} and introduction of a caching layer.

The rest of the thesis is organized as follows: the first chapter introduces the field of study, providing a basic overview of features and terminology. The second chapter presents TESLA and T-Rex, analyzing strengths and weakness. The third chapter discuss the language extension. The fourth chapter explains the architecture of the system and the algorithms used. The fifth chapter presents the result of the benchmarks and analyzes the effectiveness of the proposed solution. The last chapter draws the conclusions and proposes future development.

