\chapter*{Introduction}
\addcontentsline{toc}{chapter}{\tocEntry{Introduction}}
\markboth{INTRODUCTION}{}

% context, brief purpose and result, contributions
% detailed explanation of the contents organization

----- RISCRIVEREEEEE -----\\
The exponential growth of data production, due to the increasing spread of social platforms and \emph{Internet of Things} (IoT), has started a thriving period of research and renovation in data management technology.
The relational model, that was once considered the silver bullet of data management, has recently reached its limits of evolution and many new paradigms have been proposed and developed.

% rquire - favorire la rivoluzione - essere carburante per l'innovazione . richiedere cambiamento - ricerca, sviluppo adattamento - pushing higher the requirements - efforts - needs - raise - cause - foster - started a period of 

The way of handling data has changed dramatically - nosql - mystical creature called BigData - but also many little different requirements brought to diversification and specialization - and is still changing and increasing in volume (pervasive systems etc)
If in the past the birth of new fields led to differentiation the progressive diffusiona and stabilization is leading to overlapping, interaction and unification
As many papers about CEP will tell you, most of the systems can ultimately be modeled as event driven, however some task are still easier to face using tools and models from the tradition of DBMS, so it is iteresting the idea of joining the two.
However what are the strengths and limits of those systems and what are the consequences of their interaction and which the precaution we have to take
In this project we have the purpose to investigate such questions and answer with examples from the implementation and test of our application (some detail of what we did and what are the conclusion)
And in particular my contributions were...found ambiguities, formalized the language and proposed alternatives; modeled the interaction of the two systems and proposed a syntax (and semantic) extending tesla; reimplemented the engine to be more robust and open to extensions (the present one and all the futures) and more accurate wrt the specification; extended the system with the new features with a reference impl with sqlite and introduced a caching layer; finally tested everything to explore the effectiveness of the solution.
The rest of the thesis is organized as follows: in the first chapter an introduction to the field of study providing a context and basic overview of features and terminology; the second chapter introduces TESLA and T-Rex and analyze strengths and weakness presenting a new formalization; the third chapter will formalize the syntax and semantic extension; in the fourth chapter I will explain the architecture of the system and the algorithm used; in the fifth chapter I will present the result of the benchmarks and analyze the effectiveness of the solution proposed; in the last chapter I will draw some conclusions and propose future development

