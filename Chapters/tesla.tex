\chapter{TESLA and TRex}

\section{Introduction}

TESLA \cite{tesla} is a declarative strongly typed CEP language that, as some other alternatives, provides a comprehensive set of common operation on events (like filtering and parameterization, composition and pattern detection, negation and aggregation) and allows to control selection policies, time windows and event consumption. However, while the competitors rely on informal documentation that leaves room to ambiguities, TESLA's unprecedented characteristic is its aim to a complete semantic specification with \emph{TRIO} \cite{trio}, a first order temporal logic. The definition in advance of a precise behavior for each feature improves coherence in the development of engines based on TESLA and helps users to understand the language deeply with less empirical research. It is part of the purpose of the thesis to keep working on this track.
% TODO maybe add a rule example to give an initial hint 

TESLA reference implementation is T-Rex \cite{trex}, a CEP engine written in C++. It was initially designed around an algorithm called \emph{Automata-based Incremental Processing} (AIP), which transforms each rule in a state machine that is spawned and activated by every incoming event until successful pattern detection or failure; AIP was fundamental to analyze the complexity bounds imposed by the language. Later T-Rex has been rewritten using a different algorithm called \emph{Column-based Delayed Processing} (CDP) \cite{trex-cuda}, that works accumulating events and processing them in batch as soon as a possible trigger is detected; this technique was found to be faster and easier to parallelize and offered the opportunity for a \emph{CUDA} implementation.

While the specification of TESLA's semantic was a central topic of the very first paper, at the time of the writing the syntax was always described informally and through examples. This was identified as a risk of possible misunderstandings in the very foundation of the project, so as first thing in this chapter I will try to define the TESLA grammar in a more rigorous and hopefully clear way. Then I will summarize the previous writings about semantic and finally will highlight the inconsistencies between the semantics of TESLA and its actual implementation in the T-Rex engine, which aroused after system analysis and team discussion, followed by the clarifications or modifications proposed.

\section{BNF Grammar}

\emph{Backus-Naur Form} (BNF) is a notation for context-free grammars that allows to recursively define the composition of every single syntax feature. The components of this kind of writing are terminal symbols, that are simple strings possibly empty ($\epsilon$), and non terminal ones, that are wrapped in angle brackets. Each non terminal is defined in a derivation rule and on the right hand side there will be the non terminal name, while on the left one there will be a sequence of symbols possibly separated by a vertical line ($|$) to imply choice between options.

BNF is also used, in a machine readable form, by parser generators like \emph{ANTLR} (the one used by the project), so technically there is a definition already, but it had to go through some scarcely documented compromises to avoid parsing ambiguities and to face some practical need. The version presented here instead is meant for human comprehension and as a high level reference, so I will try make it as clear and self explanatory as possible, giving up the irrelevant details needed only for parsing purposes.

\subsection{Rule basic structure}
The outline of a rule is characterized by four main sections: definition of the derivate event, pattern of events, attribute assignment and event consumption.
\begin{bnf*}
\bnfprod{rule}{
    \bnfpn{define} \bnfsp
    \bnfpn{from} \bnfsp
    \bnfpn{where} \bnfsp
    \bnfpn{consuming}
}
\end{bnf*}

\subsection{Define clause}
The definition of a complex event is characterized by the name of the soon to be generated tuple and by a list of attributes names and types.
\begin{bnf*}
\bnfprod{define}{
    \bnfts{define} \bnfsp
    \bnfpn{capital identifier} \bnfsp
    \bnfts{(} \bnfsp
    \bnfpn{attributes} \bnfsp
    \bnfts{)}
}
\end{bnf*}
\begin{bnf*}
\bnfprod{attributes}{
    \bnfpn{attribute} \bnfsp
    \bnfpn{attributes tail} \bnfor
    \bnfes
}\\
\bnfprod{attributes tail}{
    \bnfts{,} \bnfsp
    \bnfpn{attribute} \bnfsp
    \bnfpn{attributes tail} \bnfor
    \bnfes
}\\
\bnfprod{attribute}{
    \bnfpn{lower identifier} \bnfsp
    \bnfts{:} \bnfsp
    \bnfpn{attribute type}
}\\
\bnfprod{attribute type}{
    \bnfts{int} \bnfor
    \bnfts{float} \bnfor
    \bnfts{bool} \bnfor
    \bnfts{string}
}
\end{bnf*}

\subsection{From clause}
From clause is the core of pattern detection and is made of a sequence of predicates on different events and aggregates.
\begin{bnf*}
\bnfprod{from}{
    \bnfts{from} \bnfsp
    \bnfpn{predicate body} \bnfsp
    \bnfpn{predicates}
}
\end{bnf*}
\begin{bnf*}
\bnfprod{predicates}{
    \bnfts{and} \bnfsp
    \bnfpn{predicate} \bnfsp
    \bnfpn{predicates} \bnfor
    \bnfes
}\\
\bnfprod{predicate}{
    \bnfpn{event} \bnfor
    \bnfpn{aggregate}
}
\end{bnf*}

\subsection{Where clause}
The where section is composed by a set of assignments of the attributes of the soon to be generated event using arbitrary expression over the data processed in the pattern detection phase.
\begin{bnf*}
\bnfprod{where}{
    \bnfts{where} \bnfsp
    \bnfpn{assignments} \bnfor
    \bnfes
}
\end{bnf*}
\begin{bnf*}
\bnfprod{assignments}{
    \bnfpn{assignment} \bnfsp
    \bnfpn{assignments tail}
}\\
\bnfprod{assignments tail}{
    \bnfts{,} \bnfsp
    \bnfpn{assignment} \bnfsp
    \bnfpn{assignments tail} \bnfor
    \bnfes
}\\
\bnfprod{assignment}{
    \bnfpn{lower identifier} \bnfsp
    \bnfts{=} \bnfsp
    \bnfpn{expression}
}
\end{bnf*}

\subsection{Consuming clause}
Consumption policy is simply defined as a list of names of events, that took part in the from clause. 
\begin{bnf*}
\bnfprod{consuming}{
    \bnfts{consuming} \bnfsp
    \bnfpn{capital identifier} \bnfsp
    \bnfpn{capital identifiers} \bnfor
    \bnfes
}
\end{bnf*}

\subsection{Predicate body}
The base of a predicate is made by a tuple constrained by a set of boolean expression over current event's data and parameters, possibly followed by an alias definition.
\begin{bnf*}
\bnfprod{predicate body}{
    \bnfpn{constrained tuple} \bnfsp
    \bnfpn{alias}
}\\
\bnfprod{constrained tuple}{
    \bnfpn{capital identifier} \bnfsp
    \bnfts{(} \bnfsp
    \bnfpn{constraints} \bnfsp
    \bnfts{)}
}\\
\bnfprod{constraints}{
    \bnfpn{expression} \bnfsp
    \bnfpn{constraints tail} \bnfor
    \bnfes
}\\
\bnfprod{constraints tail}{
    \bnfts{,} \bnfsp % Different from original papers!
    \bnfpn{expression} \bnfsp
    \bnfpn{constraints tail} \bnfor
    \bnfes
}\\
\bnfprod{alias}{
    \bnfts{as} \bnfsp
    \bnfpn{capital identifier} \bnfor
    \bnfes
}
\end{bnf*}
Note: in this formalization we use commas to separate constraint, opposed to $AND$ as in the original papers, to better distinguish predicates vs. filters composition.

\subsection{Event}
An event predicate (except the trigger one that we can see in the $from$ clause) adds to tuple filtering a selection policy chosen between $each, not, first, last$ and constraints about the time window.
\begin{bnf*}
\bnfprod{event}{
    \bnfpn{event selection} \bnfsp
    \bnfpn{predicate body} \bnfsp
    \bnfpn{timing}
}\\
\bnfprod{event selection}{
    \bnfts{each} \bnfor
    \bnfts{not} \bnfor % TODO maybe move in aggregates
    \bnfts{first} \bnfor
    \bnfts{last}
}
\end{bnf*}

\subsection{Aggregates}
TESLA has the common aggregators that can be found in DBMS and other CEP engines, but the list could be extended if needed. They are applied on a set of tuples filtered in a similar way to event predicates and the result can be used in an additional constraint for the event pattern.
\begin{bnf*}
\bnfprod{aggregate}{
    \bnfpn{aggregate body} \bnfsp
    \bnfpn{aggregate constraint}
}\\
\bnfprod{aggregate body}{
    \bnfpn{aggregator} \bnfsp
    \bnfts{(} \bnfsp
    \bnfpn{aggregate inner} \bnfsp
    \bnfts{)}
}\\
\bnfprod{aggregate inner}{
    \bnfpn{constrained tuple} \bnfsp
    \bnfpn{attribute selection} \bnfsp
    \bnfpn{timing} \bnfsp
}\\
\bnfprod{aggregator}{
    \bnfts{AVG} \bnfor
    \bnfts{SUM} \bnfor
    \bnfts{MAX} \bnfor
    \bnfts{MIN} \bnfor
    \bnfts{COUNT}
    % TODO maybe add EXISTS/ANY and NOT
}\\
\bnfprod{attribute selection}{
    \bnfts{.} \bnfsp
    \bnfpn{lower identifier} \bnfor
    \bnfes
}\\
\bnfprod{aggregate constraint}{
    \bnfpn{binary operator} \bnfsp
    \bnfpn{expression}
}
\end{bnf*}

\subsection{Timing}
Time constraint is imposed with two different type of window: one of given duration from a starting event, the other delimited by a couple of distinct events.
\begin{bnf*}
\bnfprod{timing}{
    \bnfpn{within} \bnfor
    \bnfpn{between}
}\\
\bnfprod{within}{
    \bnfts{within} \bnfsp
    \bnfpn{time} \bnfsp
    \bnfts{from} \bnfsp
    \bnfpn{capital identifier}
}\\
\bnfprod{between}{
    \bnfts{between} \bnfsp
    \bnfpn{capital identifier} \bnfsp
    \bnfts{and} \bnfsp
    \bnfpn{capital identifier}
}\\
\bnfprod{time}{
    \bnfpn{float} \bnfsp
    \bnfpn{time unit} \bnfsp
}\\
\bnfprod{time unit}{
    \bnfts{d} \bnfor
    \bnfts{h} \bnfor
    \bnfts{min} \bnfor
    \bnfts{s} \bnfor
    \bnfts{ms} \bnfor
    \bnfts{us}
}
\end{bnf*}

\subsection{Expressions and constraints}
Expressions in TESLA are common algebraic, boolean and string operations composed with each other and they can operate on immediate values, references to current tuple attributes or parameters.
\begin{bnf*}
\bnfprod{expression}{
    \bnfpn{parenthesization} \bnfor
    \bnfpn{operation} \bnfor
	\bnfpn{atom}
}\\
\bnfprod{parenthesization}{
	\bnfts{(} \bnfsp
    \bnfpn{expression} \bnfsp
    \bnfts{)}
}\\
\bnfprod{operation}{
    \bnfpn{binary operation} \bnfor
    \bnfpn{unary operation}
}\\
\bnfprod{binary operation}{
    \bnfpn{expression} \bnfsp
    \bnfpn{binary operator} \bnfsp
    \bnfpn{expression}
}\\
\bnfprod{unary operation}{
    \bnfpn{unary operator} \bnfsp
    \bnfpn{expression}
}\\
\bnfprod{binary operator}{
    \bnftd{Common algebraic and comparison operators}
}\\
\bnfprod{unary operator}{
    \bnftd{Common unary operators}
}\\
\bnfprod{atom}{
    \bnfpn{identifier} \bnfor
    \bnfpn{parameter} \bnfor
    \bnfpn{immediate}
}\\
\bnfprod{identifier}{
	\bnfpn{qualifier} \bnfsp
    \bnfpn{lower identifier}
}\\
\bnfprod{qualifier}{
    \bnfpn{capital identifier} \bnfsp
    \bnfts{.} \bnfor
    \bnfes
}
\end{bnf*}

\subsection{Basic types}
These basic symbols help defining the rest of the BNF. We can notice the three types of identifiers: $capital$ used for event names, $lower$ used for tuple attributes and $parameter$ obviously used for parameterization.
\begin{bnf*}
\bnfprod{capital identifier}{
    \bnftd{An identifier starting with an uppercase letter}
}\\
\bnfprod{lower identifier}{
    \bnftd{An identifier starting with a lowercase letter}
}\\
\bnfprod{parameter}{
    \bnfts{\$} \bnfsp
    \bnfpn{lower identifier}
}\\
\bnfprod{capital identifiers}{
    \bnfts{,} \bnfsp
    \bnfpn{capital identifier} \bnfsp
    \bnfpn{capital identifiers} \bnfor
    \bnfes
}\\
\bnfprod{lower identifiers}{
    \bnfts{,} \bnfsp
    \bnfpn{lower identifier} \bnfsp
    \bnfpn{lower identifiers} \bnfor
    \bnfes
}\\
\bnfprod{immediate}{
    \bnftd{An immediate value, like a digit}
}\\
\bnfprod{float}{
    \bnftd{A floating point number}
}
\end{bnf*}

\section{Semantic}
% TODO maybe a small intro?

\subsection{TRIO overview}
Created especially to address the needs of real-time systems specification, TRIO \cite{trio} is a formalism based on the extension of \emph{First Order Logic} (FOL) with temporal operators that allow to describe properties in evolution and even to express distance and duration in time. This ability to reason quantitatively makes TRIO different from classical temporal logics and particularly suitable to deal with events and their occurrences.

The syntax is composed of the typical elements of a FOL, like variables, functions, predicates, propositional operators and quantifiers. In addition to those there are special temporal operators $Futr(A, t)$ and $Past(A, t)$ (plus derivatives). Moreover usual arithmetic functions, like $+$ and $-$, and common relational predicates, like $=$ and $<$, are assumed to be predefined at least for the temporal domain.

The semantic is again based on FOL, but some variables and predicates are time-dependent, meaning that their value changes in different instants. In the interpretation of a formula there is a present time, left implicit, that is the reference point for any temporal expression. The operator $Futr(A, t)$ is $true$ if $A$ holds $t$ time units in the future and respectively $Past(A, t)$ is $true$ if $A$ holds $t$ time units in the past.\\
From those basic operators many others can be derived, but for the purpose of the thesis we will need just two of them:
\begin{align*}
&Alw(A) = A \wedge\ \forall t > 0\ Futr(A,t) \wedge\ \forall t > 0\ Past(A, t)\\
&WithinP(A, t_1, t_2) = \exists x\ (t_1 \le x \le t_1 + t_2 \wedge Past(A,x))
\end{align*}
$Alw(A)$ intuitively means that $A$ is always $true$ in the past, present and future. While $WithinP(A, t_1, t_2)$ means that $A$ is valid in some past instant within $t_1$ and $t_1 + t_2$.

\subsection{Basic concepts}
\subsubsection{Labels and uniqueness of selection}
First of all we have to introduce the concept of $label$: a unique global identifier, that makes possible to discern one event from another, even when they share the very same attributes and time. We assume that incoming events have been correctly labeled before entering the system, while to determine the label of a generated complex event we define the function $lab(r, s)$, where $r$ is the rule that was triggered, $s$ is the list of labels of the events that satisfied the pattern and the returned value is the label for the emitted event.\\
It is fundamental that $lab$ will always generate a different label for each new event, so the function must be injective:
\begin{align*}
Alw(\forall\ r_1,s_1,r_2,s_2\ ((lab(r_1, s_1) = lab(r_2, s_2)) \leftrightarrow (r_1=r_2 \wedge s_1=s_2)))
\end{align*}
At the same time we need to ensure the $uniqueness\ of\ selection$, which states that a rule $r$ can be applied to the same list of events $s$ only once. We notice that there is always a trigger event that happen at the same time of the generated one, so in a new time instant the trigger is necessarily changed. While during a single instant it's meaningless to generate more than one event from the very same list.

\subsubsection{Event occurence}
To describe the occurrence of an event we introduce the time-dependent predicate $Occurs(e, l)$, where $e$ is the type of the event and $l$ is the label, the predicate is $true$ in the single instant of occurrence and $false$ in any other.
\begin{align*}
Alw(\forall e_1, e_2 \in E, \forall l \in L\ (Occurs(e_1, l) \wedge Occurs(e_2, l) \rightarrow e_1 = e_2))
\end{align*}
\begin{align*}
  &Alw(\forall e_1, e_2 \in E, \forall l \in L, \forall t > 0\ (Occurs(e_1, l) \rightarrow\\
  &\neg Past(Occurs(e_2, l), t) \wedge \neg Futr(Occurs(e_2, l), t)))
\end{align*}
Where E is the set of event types and L the set of valid labels.

\subsubsection{Time of occurence}
We define a time-dependent function $time(l)$, where $l$ is a label of an event, that returns (if any) the time of occurrence in the past with respect to the current instant.
\begin{align*}
  &Alw(\forall l \in L, \forall t > 0\ (time(l) = t \leftrightarrow Past(Occur(l), t)))
\end{align*}

\subsubsection{Attributes values}
Finally to reason about the content of an event we define the function $attVal(l, n)$, where $l$ is a label, $n$ is a valid attribute name and the result is the value of the attribute. \footnote{The concept is here simplified and doesn't take into account attributes types and domains, but for explanation purposes it is the right level of abstraction.}

\subsection{Specification}
% TODO maybe write a small intro

\subsubsection{Event emission}
As introduced during the syntax description, the $define$ statement declares the event to be generated and the rule is triggered as soon as a sequence of events satisfies the pattern of $from$ section.\\
The simplest rule is the one that produces a complex event $CE$ for any notification of simple event $SE$.
\begin{align*}
&define\ CE()\ from\ SE()\ \triangleq
\end{align*}
\begin{align*}
&Alw(\forall l(Occurs(CE,\ lab(\{l\})) \leftrightarrow Occurs(SE,\ l)))
\end{align*}
The next step is being able to filter the trigger event, using basic operations on its attributes. So that $CE$ is generated if and only if $SE$'s attributes matches the all the constraints.
\begin{align*}
&define\ CE()\ from\ SE(att_1\ op_1\ val_1,\ ...,\ att_n\ op_n\ val_n)\ \triangleq
\end{align*}
\begin{align*}
&Alw(\forall l(Occurs(CE,\ lab(\{l\})) \leftrightarrow (Occurs(SE,\ l)\ \wedge\\
&attVal(l,\ att_1)\ op_1\ val_1 \wedge\ ...\ \wedge attVal(l,\ att_n)\ op_n\ val_n)))
\end{align*}
The constraints can be generalized from a single operator to generic expressions over tuple attributes and, as we will see, parameters. Each expression is mapped over a corresponding TRIO predicate that depends on the same arguments.
\begin{align*}
&define\ CE()\ from\\
&SE(expr_1(att_1,\ ...,\ att_n),\ ...,\ expr_m(att_1,\ ...,\ att_n))\ \triangleq
\end{align*}
\begin{align*}
&Alw(\forall l (Occurs(CE,\ lab(\{l\})) \leftrightarrow (Occurs(SE,\ l) \wedge\\
&Pred_1(attVal(l,\ att_1),\ ...,\ attVal(l,\ att_n)) \wedge\ ...\ \wedge\\
&Pred_m(attVal(l,\ att_1),\ ...,\ attVal(l,\ att_n)))))
\end{align*}
To assign values to $CE$ attributes we use the $where$ clause and each TESLA expressions is mapped to a TRIO function $f_i$.
\begin{align*}
&define\ CE(att_1,\ ...,\ att_n)\ from\ SE()\\
&where\ att_1 = expr_1(SE.att_1,\ ...,\ SE.att_m),\ ...,\\
&att_n = expr_n(SE.att_1,\ ...,\ SE.att_m)\ \triangleq
\end{align*}
\begin{align*}
&Alw(\forall l ((Occurs(CE,\ lab(\{l\})) \leftrightarrow Occurs(SE,\ l))\\
&\wedge (Occurs(CE,\ lab(\{l\})) \rightarrow\ attVal(lab(\{l\}),\ att_1) =\\
&f_1(attVal(l,\ att_1),\ ...,\ attVal(l,\ att_m)) \wedge\ ...\ \wedge\\
&attVal(lab(\{l\}),\ att_n) = f_n(attVal(l,\ att_1),\ ...,\ attVal(l,\ att_m)))))
\end{align*}

\subsubsection{Event composition}
To capture patterns of multiple events, TESLA provides some composition operators, made of a selection policy ($each$, $first$, $last$, $not$) and a window ($within$ or $between$).\\
The operator $within$ defines a time range that goes from the occurrence of a specified event to a fixed amount of time in the past. The operator $between$ defines a time range delimited by two different events.\\
The selection $each$ emits an event for every occurrence inside the specified window.
\begin{align*}
&define\ CE()\ from\ A()\ and\ each\ B()\ within\ x\ from\ A\ \ \triangleq
\end{align*}
\begin{align*}
&Alw(\forall l_0, l_1 \in L\ (Occurs(CE, lab(r, \{l_0, l_1\}))\ \leftrightarrow\\
&(Occurs(A, l_0)\ \wedge\ WithinP(Occurs(B, l_1),\ time(l_0),\ x))))
\end{align*}
The selection $last$ emits a single event using the most recent occurrence in the window (ties are broken using an artificial total ordering on labels).
\begin{align*}
&define\ CE()\ from\ A()\ and\ last\ B()\ within\ x\ from\ A\ \ \triangleq
\end{align*}
\begin{align*}
&Alw(\forall l_0, l_1 \in L\ (Occurs(CE, lab(r, \{l_0, l_1\}))\ \leftrightarrow\\
&(Occurs(A, l_0)\ \wedge\ WithinP(Occurs(B, l_1),time(l_0),x)\\
&\wedge\ \neg (\exists l_2 \in L, \exists t\in [time(l_0), time(l_1))\ Past(Occurs(B, l_2),t))\\
&\wedge\ \neg \exists l_3 \in L\ (Past(Occurs(B, l_3), time(l_1))\ \wedge\ l_3 > l_1)))) 
\end{align*}
The selection $first$ emits a single event using the least recent occurrence in the window (ties are broken using an artificial total ordering on labels).
\begin{align*}
&define\ CE()\ from\ A()\ and\ first\ B()\ within\ x\ from\ A\ \ \triangleq
\end{align*}
\begin{align*}
&Alw(\forall l_0, l_1 \in L\ (Occurs(CE, lab(r, \{l_0, l_1\}))\ \leftrightarrow\\
&(Occurs(A, l_0)\ \wedge\ WithinP(Occurs(B, l_1),time(l_0),x)\\
&\wedge\ \neg (\exists l_2 \in L, \exists t\in (time(l_1), time(l_0) + x]\ Past(Occurs(B, l_2),t))\\
&\wedge\ \neg \exists l_3 \in L\ (Past(Occurs(B, l_3), time(l_1))\ \wedge\ l_3 < l_1))))
\end{align*}
The negation emits an event if there aren't occurrences in the window.
\begin{align*}
&define\ CE()\ from\ A()\ and\ not\ B()\ within\ x\ from\ A\ \ \triangleq
\end{align*}
\begin{align*}
&Alw(\forall l_0, l_1 \in L\ (Occurs(CE, lab(r, \{l_0, l_1\}))\ \leftrightarrow\\
&(Occurs(A, l_0)\ \wedge\ \neg WithinP(Occurs(B, l_1),\ time(l_0),\ x))))
\end{align*}
The same principles apply with $between$ operator:
\begin{align*}
&define\ CE()\ from\ A()\ and\ each\ B()\ within\ x\ from\ A\\
&and\ each\ C()\ between\ A\ and\ B\ \triangleq
\end{align*}
\begin{align*}
&Alw(\forall l_0, l_1, l_2 \in L\ (Occurs(CE, lab(r, \{l_0, l_1, l_2\}))\ \leftrightarrow\\
&(Occurs(A, l_0)\ \wedge\ WithinP(Occurs(B, l_1),\ time(l_0),\ x)\\
&\wedge\ WithinP(Occurs(C, l_2),\ time(l_0),\ time(l_1) - time(l_0)))))
\end{align*}

\subsubsection{Parameterization}
Parameters, identified by the leading dollar sign, work as variables to interconnect filter expressions in different predicates inside a rule. In TRIO they can be represented simply by adding an additional constraint to the formula.
\begin{align*}
&define\ CE()\ from\ A(att_0 = \$x)\ and\\
&each\ B(att_2 < \$x)\ within\ x\ from\ A\ \ \triangleq
\end{align*}
\begin{align*}
&Alw(\forall l_0, l_1 \in L\ (Occurs(CE, lab(r, \{l_0, l_1\}))\ \leftrightarrow\\
&(Occurs(A, l_0)\ \wedge\ WithinP(Occurs(B, l_1),\ time(l_0),\ x)\\
&\wedge attVal(l_0, att_0) > attVal(l_1, att_1))))
\end{align*}

\subsubsection{Aggregates}
TESLA provides the most common aggregates function, that compute a single result from a set of events in a window. They can be used for filtering and for value assignment.
\begin{align*}
&AGGR(A.attr\ within\ x\ from\ B) = y\ \ \triangleq
\end{align*}
\begin{align*}
&Alw(\forall S (\forall s (s \in S\ \leftrightarrow \exists l \in
L (s = \langle l,\ attVal(l,attr) \rangle\\
&\wedge\ WithinP(Occurs(A,l),\ Time(B),\ x))) \rightarrow\ aggr(S)=y))
\end{align*}

\subsubsection{Event consumption}
Event consumption is the ability to mark an event that previously matched in a pattern to be removed from the list of the one available to trigger the rule again. To describe this behavior in TRIO we introduce the predicate $Consumed(r, l)$, that should be $false$ until the rule $r$ consumes the event labeled $l$ and remain $true$ from then on.
\begin{align*}
&Alw(\forall l \in L, \forall r \in R (Consumed(r, l) \rightarrow\\
&\forall\ t > 0\ Futr(Consumed(r, l),\ t)))
\end{align*}
\begin{align*}
&Alw(\forall l \in L, \forall e \in E, \forall r \in R, \forall S\ ((\neg \exists t > 0\ (l \in S\ \wedge\\
&Past(Occurs(e, lab(r, S),t)))) \rightarrow \neg Consumed(r, l)))
\end{align*}
We can use this predicate to describe rules with consuming clauses.
\begin{align*}
&define\ CE()\ from\ A()\ and\ each\ B()\\
&within\ x\ from\ A\ consuming\ B\ \triangleq
\end{align*}
\begin{align*}
&Alw(\forall l_0, l_1 (Occurs(CE, lab(\{l_0, l_1\})) \leftrightarrow Occurs(A, l_0) \wedge\\
&WithinP(Occurs(B, l_1), time(l_0), x) \wedge \neg Consumed(r, l_1)\\
&Occurs(CE, lab(\{l_0, l_1\})) \rightarrow \forall t > 0\ Futr(Consumed(r, l_1), t)))
\end{align*}

\subsubsection{Hierarchies and iterations}
Unlike other CEP languages, TESLA allows complex events, defined through rules, to be part of other detection patterns and not only to be used as subscription topics.

The straight forward consequence is the possibility to create a hierarchical structure. The composition improves code organization, reuse and maintainability, which are fundamental goals of any programming language, but it also makes possible progressive data refinement, that eases development in contexts characterized by different layers of information detail.

At the same time, complex events can be referenced in a recursive approach, that gives the power to express iterations without any additional language constructs. Iteration is necessary to analyze the evolution in time of the input flows and detect trends that could span over an undefined number of events.
\begin{align*}
&define\ RepA(Times, Val)\ from\ A()\\
&where\ Times=1\ and\ Val=A.Val\\
\\
&define\ RepA(Times, Val)\ from\ A(\$x)\\
&and\ last\ RepA(Val \le \$x)\ within\ 3 min\ from\ A\\
&where\ Times=RepA.Times+1\ and\ Val = \$x\\
&consuming\ RepA\\\\
&define\ B(Times)\ from\ RepA()\\
&where\ Times=RepA.Times
\end{align*}

This ability to create loops is powerful and tunable, but introduces the risk of non-termination typical of highly expressive languages.

\section{Ambiguities and clarifications}

During the analysis of the previously defined syntax, it appeared clear the presence of some ambiguities in the semantics of some specific TESLA operators and that, besides the additions of static data, a general renewal of the language was appropriate.

\subsection{Parameters}
In previous papers, parameterization was presented only through examples and there was no explicit definition of its expressivity. So it wasn't clear if parameters needed two separate phases of assignment and usage or they could appear as free variables in high level logical constraints.\\
The former option appeared to be the intended one and few more issues followed from that consideration, for example it wasn't stated if the order of definition and usage was relevant, there was no difference between assignment and comparison operators and parameters definition was mixed with selection predicates, without any syntax distinction.\\
We need the syntax to make everything more clear and intuitive by reordering clauses and separating parameters definitions.

\subsection{Event declaration}
TESLA was presented as a strongly typed language and so it was expected to prevent type incoherences before the actual rule evaluation. However simple events weren't declared beforehand and just appeared at runtime as input to the system, while complex event could be emitted by different rules possibly with different signatures. That lack of information made hard to check in advance.\\
More over it was impossible to assign a numerical id to an event type for more convenient interaction with external systems.\\
Some of these issues have been tackled informally in the implementation, but the solution adopted looked more like an unavoidable patch rather than a design choice. To fill those gaps we propose a new independent statement to declare the signature and the id of any event that is going to be processed by the system.

\subsection{Filtering}
The use of the keyword WHERE for attributes assignment can be counterintuitive, since it wrongly reminds of the SQL  clause for rows filtering.\\
Moreover filtering was made in the FROM clause, mixed with the selection predicates, in a way that could easily became messy.\\
The language revision should improve keywords choice to facilitate the approach of newcomers and it should better separate selection and filtering to keep statements as clean as possible.

\subsection{Timestamp and selection}
During event composition, inclusion or exclusion of the present time instant can make a great difference, but the topic wasn't clearly discussed in previous writings.

In the original paper the predicate $WithinP(A, t_1, t_2)$ was defined using the symbol $\le$, so the instants $t_1$ and $t_2$ were considered part of the range. However motivations and consequences of this decision weren't explained.\\
The exclusion of the present instant wouldn't allow hierarchical and recursive composition, which are needed to modularize rules when complexity increases and to have enough expressivity power to describe iterations.\\
Nevertheless the choice of closed ranges, which provides that kind of expressivity, bring some downsides too. For example it is possible to create paradoxes with events that should be generated if they are not and shouldn't be generated if they are, like:
\begin{align*}
define\ A\ from\ B\ and\ not\ A\ within\ X\ from\ B
\end{align*}
At the same time it is impossible to write a rule like:
\begin{align*}
define\ B\ from\ A\ as\ A1\ and\ not\ A\ within\ X\ from\ A1
\end{align*}
that sounds reasonable at a first look, but it wouldn't emit any event.

A practical solution is the one used by the reference implementation that breaks paradoxes introducing an implicit total ordering: it evaluates conditions in execution order and never looks back. The price paid is a runtime nondeterminism, which means that different processing order of logically concurrent events could lead to different outcomes.\\
This approach is the most reasonable found so far and should be considered the standard, but the users should be aware of the implications and topic should remain open to further discussion.

\subsection{Consuming}
% TODO maybe write about incoherences between paper and implementation
Event consumption was always described in combination with $each$ selection policy, while the use of $first$ or $last$ wasn't really clarified. In particular, when the first (resp. last) event satisfying the pattern was already marked as consumed, it wasn't specified if the match should be considered terminated, without complex event emission, or the processor should just look for the next one that is not consumed. The reference implementation adopted the latter approach and we decided to take it as the standard.