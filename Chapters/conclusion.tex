\chapter*{Conclusions}
\addcontentsline{toc}{chapter}{\tocEntry{Conclusions}}
\markboth{CONCLUSIONS}{}

% TODO state of the art? (spark? flink? esper?)

In this dissertation we presented an extension to the TESLA language to describe the interaction between persistent data collection and event streams, showing that is possible to obtain a powerful abstraction without loosing the simplicity of the declarative language and preserving the formal definition of its operators.

We showed how, with an embedded database, its possible to handle millions of rows within the strict latency requirements of a real-time software and how, in presence of discrete and recurring data, the use of a cache can bring the performances almost on par with those of an equivalent load of pure events. In particular we demonstrated that the integration developed outperforms the workarounds that were previously needed to emulate the functionality.

However we verified that the attached source of data has to reliably respond within acceptable time constraints, otherwise not even the best cache can compensate for the slowdown. The problem is emphasized by the guaranties of time order execution enforced with a blocking architecture: this has been found to be a risky point of failure, because a delay of even a single call to the database may cause the system to stall. So for future development we suggest the investigation of a fine grained model of concurrency, the possibility to configure the desired level of guaranties and a integrated mechanism of error handling.

We also observed that the language allows multiple formulation of equivalent rules which may produce different execution plans, at the moment the responsibility to choose the most efficient alternative is left to the programmer, but this task could be automated introducing a step of rule rewriting.

In conclusion we think that the extension was a successful example of the proposed interoperabiliy and it is worth refining and expanding with new adapters. However it introduced many changes to the project and in the immediate future there will be a period of stabilization and reintegration of all the existing components, like the parser and the gpu processor.