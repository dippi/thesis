\chapter*{Conclusions}
\addcontentsline{toc}{chapter}{\tocEntry{Conclusions}}
\markboth{CONCLUSIONS}{}

% TODO state of the art? (spark? flink? esper?)

In this dissertation we presented an extension to the TESLA language to describe the interaction between static data collections and event streams, showing that is possible to have a powerful abstraction without compromising the simple and declarative nature of TESLA language or its formal definition.

We showed how the T-Rex engine, operating in connection with an SQLite database, can handle queries over tables of millions of rows within the strict latency requirements of a real-time system and how, in presence of discrete and recurring query parameters, the use of a cache can make the engine perform almost as fast as if it was handling a similar load of pure events. In particular we demonstrated that the component developed outperforms the workarounds that were previously needed to emulate the access to persistent information.

However we verified that the response time of the database, even with the use of caches, has a critical impact on the entire system and the problem is emphasized by the guaranties of time order execution enforced in T-Rex with a blocking architecture. This has been found to be a risky point of failure, because a delay of even a single call to the database may cause the engine to stall. So for future development we suggest the investigation of a fine grained model of concurrency and the possibility to configure the desired level of guaranties.

We also observed that the TESLA language allows multiple formulation of equivalent rules, which may produce different execution plans. At the moment the responsibility of choosing the most efficient alternative is left to the programmer, but this task could be automated introducing a step of rule rewriting.

In conclusion we think that the integration of SQLite into the T-Rex engine was a successful example of the proposed interoperabiliy between CEP tools and DBMSs. So we believe that the presented implementation is worth expanding with new adapters for other data sources. However, as for now, the introduction of so many changes to the project will require a period of stabilization and reintegration of all the existing modules, like the parser and the gpu processor.