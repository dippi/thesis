%*******************************************************
% Abstract
%*******************************************************
\chapter*{Abstract}
\addcontentsline{toc}{chapter}{\tocEntry{Abstract}}
\markboth{ABSTRACT}{}

Data management systems over the years developed in two diametrical opposite fields of application: data storage and stream processing.\\
Nowadays both approaches are wide spread and they are often required to cooperate toward common goals. However the integration of the two is still in an early stage of development and customized solutions for each specific deployment are usually required.

The purpose of this thesis is to extend the TRex \emph{Complex Event Processing} engine \cite{trex} to allow a native and general purpose interoperability with relational databases and static data sources in general.\\
The task will be studied starting from the language design and extension, with the goal of being a natural evolution of the current expressivity. Then an implementation will be presented, explaining the choice made and the key points of the processing strategies. Finally the results of performance testing will be analyzed in relation to the requirements typical of a real-time software.

\chapter*{Sommario}
\addcontentsline{toc}{chapter}{\tocEntry{Sommario}}
\markboth{SOMMARIO}{}
I sistemi di data management nel corso degli anni si sono sviluppati in due campi di applicazione diametralmente opposti: immagazzinamento di informazioni e processazione di stream.\\
Al giorno d'oggi entrambi gli approcci sono ampiamente diffusi e spesso è necessario che cooperino per il raggiungimento di comuni obiettivi. Tuttavia l'integrazione dei due è ancora in una fase iniziale di sviluppo e di solito sono necessarie soluzioni personalizzate allo specifico caso d'uso.

Lo scopo della tesi è di estendere il software di \emph{Complex Event Processing} TRex \cite{trex} per permettere una naturale e generalizzata interoperabilità con i database relazionali e altre sorgenti di dati statici.\\
Il procedimento verrà studiato a partire dalla modellazione ed estensione del linguaggio, con l'obbiettivo di essere una naturale evoluzione dell'espressività attuale. A seguito verrà presentata un'implementazione, motivando le scelte fatte e i punti chiave del processo di esecuzione. In fine verranno analizzati i risultati di test di performance, in relazione con i tipici requisiti di un sistema real-time.